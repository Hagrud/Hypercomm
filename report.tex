\documentclass[10pt,a4paper]{article}
\usepackage[utf8]{inputenc}
\usepackage[T1]{fontenc}
\usepackage[french]{babel}

\usepackage{amsmath}
\usepackage{amssymb}
\usepackage{amsfonts}
\usepackage{stmaryrd}

\usepackage{tikz}
\usetikzlibrary{arrows,automata}


\title{Distributed systems : Building~a~distributed~data-manager}
\author{Polet Pierre-Etienne}
\date{2017, second semester}

{\large

\begin{document}
	\maketitle
	
	\section*{Introduction:}
		For this project we had to build a distributed data-manager.
		
	\section{Topology choice:}
		I choose to use an hypercube topology for this project, because of the main advantages of this topology:
		\begin{itemize}
			\item Each nodes are at distance at most $\log_2(N)$ from the others, so we can expect a low number of messages to transmit informations through the network.
			\item Each node have $\log_2(N)$ neighbors, so we hope to prevent to get a non-connected network in case of crash of few of them.
		\end{itemize}
		
		Intuitively this topology may allow to have a low numbers of messages and a "safe" network without getting an highly connected graph. On the other hand there is no order between the nodes and solutions to communicate efficiently seems not trivial.
		
		\subsection{Node definition:}
			In the topology a node will be an Erlang thread, once assigned in the network each node will know its unique identifier related to this network.
			
			The identifier ID of a node is an integer, each node will be connected to another node if and only if their ID differ from 1 digit in their binary representation.
			
		\subsection{Build the network:}
			The base structure is a doubly linked topology, we used those links only to detect the loss of a node and/or construct the network.
			
			Then for "real" communication, each nodes will get a link to nodes with one different bit in their Id. (See  Figures 1 and 2).
		
			\begin{figure}
							\begin{tikzpicture}[-,>=stealth',shorten >=1pt,auto,node distance=2cm,
							semithick]
							\tikzstyle{every state}=[fill=white,draw=blue,text=black]
							
							\node[state] (0)                 {$0$};
							\node[state] (1) [right of=0] 	{$1$};
							\node[state] (2) [right of=1] 	{$2$};
							\node[state] (3) [right of=2] 	{$3$};
							\node[state] (4) [right of=3] 	{$4$};
							\node[state] (5) [right of=4] 	{$5$};
							\node[state] (6) [right of=5] 	{$6$};
							\node[state] (7) [right of=6] 	{$7$};
							
							\path 
							(0) edge[bend left] 	node {} (1)
							(0)	edge[bend left]		node {} (2)
							(0)	edge[bend left]		node {}	(4)
							
							%(1) edge[->] 			node {} (2)
							(1) edge[bend right]	node {} (3)
							(1) edge[bend right]	node {} (5)
							
							(2) edge[bend left]		node {} (3)
							(2) edge[bend left]		node {} (6)
							
							%(3) edge[->] 	 		node {} (4)
							(3) edge[bend right]	node {} (7)
							
							(4) edge[bend left]		node {} (5)
							(4) edge[bend left]		node {} (6)
							
							%(5) edge[->]  	 		node {} (6)
							(5) edge[bend right]	node {} (7)
							
							(6) edge[bend left]		node {} (7);
							
							\end{tikzpicture}
				\caption{Links in an hypercube topology. (communication)}
			\end{figure}
			
			\begin{figure}
				\begin{tikzpicture}[-,>=stealth',shorten >=1pt,auto,node distance=2cm,semithick]
					\tikzstyle{every state}=[fill=white,draw=blue,text=black]
							
					\node[state] (0)                 {$0$};
					\node[state] (1) [right of=0] 	{$1$};
					\node[state] (2) [right of=1] 	{$2$};
					\node[state] (3) [right of=2] 	{$3$};
					\node[state] (4) [right of=3] 	{$4$};
					\node[state] (5) [right of=4] 	{$5$};
					\node[state] (6) [right of=5] 	{$6$};
					\node[state] (7) [right of=6] 	{$7$};
							
					\path
					(0) edge[->] 			node {} (1)
					(0) edge[bend left] 	node {} (1)
					(0)	edge[bend left]		node {} (2)
					(0)	edge[bend left]		node {}	(4)
							
					(1) edge[->] 			node {} (2)
					(1) edge[bend right]	node {} (3)
					(1) edge[bend right]	node {} (5)
					
					(2) edge[->] 			node {} (3)	
					(2) edge[bend left]		node {} (3)
					(2) edge[bend left]		node {} (6)
							
					(3) edge[->] 	 		node {} (4)
					(3) edge[bend right]	node {} (7)
					
					(4) edge[->] 			node {} (5)	
					(4) edge[bend left]		node {} (5)
					(4) edge[bend left]		node {} (6)
							
					(5) edge[->]  	 		node {} (6)
					(5) edge[bend right]	node {} (7)
					
					(6) edge[->] 			node {} (7)	
					(6) edge[bend left]		node {} (7);
							
				\end{tikzpicture}
				\caption{Links in our topology. (Complete topology)}
			\end{figure}
			
			We define two way of building this topology:
			
			\begin{itemize}
				\item Using only communication structure, with a wave algorithm we search the node which have the minimum number of neighbors and he add the new node.
				
				But:
				\begin{itemize}
					\item if we loss node in the "middle" of the topology we may not quickly replace them.
					\item it's not easy to deal with the loss of a node during the wave algorithm.
				\end{itemize}
				
				
				\item With the "debug" links we send the new node to the next node in the topology until we found one without next node.
				
				But:
				\begin{itemize}
					\item This algorithm do not take advantage of the topology.
				\end{itemize}
			\end{itemize}

				
				
		\subsection{Basic communication algorithm:}
			
			\paragraph{Send to ID:}
			
			\paragraph{Broadcast:}
			
\end{document}
}